\documentclass{article}

\usepackage[italian]{babel}     %testi autogenerati italiano
\usepackage{minted}             %per codice vhdl bello
\usepackage{tikz}               %per disegno fsm
\usetikzlibrary{automata, positioning, arrows}
\usepackage{circuitikz}         %per disegno componente
\usepackage{graphicx}           %per importare immagini
\usepackage{geometry}           %per gestire margini e spostamenti
\geometry {
    top=20mm,
    bmargin=20mm,
}
\usepackage{array}              %per colonne di width fissata
\usepackage{subcaption}         %tabelle divise
\usepackage{hyperref}           %links
\hypersetup{
    colorlinks=true,
    linkcolor=black,
    urlcolor=blue
}
%\usepackage[bottom]{footmisc}   %footnotes fissate a piè pagina
\usepackage{booktabs}           %per tabitem in tabular
\newcommand{\tabitem}{~~\llap{\textbullet}~~}
\renewcommand*{\thefootnote}{[\arabic{footnote}]}

\begin{document}

\setlength\parindent{0pt} %noindent automatico
\setlength\parskip{1em}

\begin{titlepage}
    \centering
    \hrule

    \vspace{0,5cm}
    {
        \normalsize Politecnico di Milano\\
        Dipartimento di Elettronica, Informazione e Bioingegneria
    }

    \vspace{5cm}
    {\Huge \textbf{Progetto di Reti Logiche\\
            2021/22}\\}

    \vspace{0,5cm}
    \large {Prof. Palermo Gianluca}

    \vspace{5cm}
    {
        \large
        \begin{tabular}{c c}
            Dario Simoni \\
            (Codice Persona: 10697990, Matricola: 932957) \\
        \end{tabular}

    }

    \vspace{6.5cm}


    \hrule

\end{titlepage}

\pagebreak

\tableofcontents

\pagebreak

\section{Requisiti di progetto} %1
\subsection{Descrizione del problema} %1.1
Si vuole realizzare un modulo HW (descritto in VHDL) che si interfacci con una memoria e che
ricevendo in ingresso un flusso continuo X da 1 bit restituisca un nuovo flusso continuo Y da 1 bit.
\par
Il flusso continuo X è generato da una seguenza continua di parole serializzate in base alla specifica, mentre il flusso continuo Y restituisce il doppio dei bit rispetto al flusso X.
\vspace

\clearpage

\end{document}